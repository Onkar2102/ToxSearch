\section{Conclusion}

This work introduced Speciated ToxSearch, a quality-diversity approach to adversarial prompt discovery that partitions the solution space into behaviorally distinct species. Our experiments demonstrate two primary findings. First, speciated evolutionary search achieves higher peak toxicity than baseline search (maximum toxicity 0.73 vs.\ 0.47) while simultaneously exploring a broader semantic landscape—evidenced by higher effective topic diversity ($N_1$) and greater unique topic coverage ($K$). Although both approaches discover comparable proportions of moderately toxic prompts (95th percentile $\approx$ 0.30), speciated search finds substantially more extreme cases, with a top-10 median toxicity of 0.66 compared to 0.44 for baseline. Second, speciation yields well-separated clusters (mean separation ratio $\approx$ 1.9) where each species exhibits a distinct toxicity distribution. The top-performing species achieve consistent high-toxicity outputs (max $\approx$ 0.7), while lower-performing species explore different regions of the toxicity landscape. This confirms that speciation successfully partitions prompts into semantically and behaviorally distinct niches, each characterized by its own toxicity profile.

Several limitations should be acknowledged. Our evaluation relies on a single toxicity oracle (Perspective API), and results may vary with alternative classifiers or human annotation. The experiments were conducted with five runs per condition using a fixed prompt-generator and response-generator pairing, limiting generalizability across model architectures. Additionally, the current speciation implementation introduces constraints in species capacity and stagnation handling that may affect some species-level interpretations; while species are well-separated overall, occasional label inconsistencies warrant further investigation.

Future work could address these limitations through multi-model evaluation across diverse toxicity classifiers and target LLMs. Incorporating alternative or learned behavior descriptors—beyond the current embedding-based distance—may improve species boundaries. Stronger QD baselines such as MAP-Elites or Rainbow-style archives would provide more rigorous comparisons. Expanding the number of runs and seeds would strengthen statistical power, while refining speciation mechanics (capacity enforcement, stagnation triggers, label assignment) could improve cluster stability. Finally, human-in-the-loop validation of semantic clusters would verify that discovered species correspond to meaningfully distinct categories of harmful content, enabling more targeted red-teaming efforts.
